\clearpage
\phantomsection

\addcontentsline{toc}{chapter}{Tóm tắt}
\chapter*{\fontsize{13}{13}\selectfont{Tóm tắt}}
\fontsize{12}{12}\selectfont{
\noindent\textbf{Tóm tắt:}
Deep learning (DL), hay học sâu, là một nhánh của machine learning (ML), hay học máy, và trí tuệ nhân tạo (AI), hiện nay được coi là một công nghệ hết sức quan trọng và tiềm năng trong thời đại công nghiệp 4.0. Mạng neuron nhân tạo được ra đời lấy ý tưởng từ chính mạng neuron trong não bộ con người qua việc xử lý dữ liệu qua nhiều tầng giúp làm rõ nghĩa của các loại dữ liệu, qua đó làm nền tảng cho Deep learning. Deep learning hiện nay được ứng dụng trong nhiều lĩnh vực như xe tự lái, robot, xử lý ngôn ngữ, chatbot, ... \\
Trong số các ứng dụng của Deep learning vào thời gian vài năm trở lại đây, có một chủ đề rất được chú ý đó là nhận diện vật thể hay Object Detection. Các thuật toán mới về nhận diện vật thể như YOLO có tốc độ thực hiện tương đối nhanh và có thể áp dụng trong các hệ thống thời gian thực như kết hợp vào SLAM hay xe tự lái để định vị. \\
Đồ án này sẽ tập trung vào một thuật toán state-of-the-art trong lĩnh vực nhận diện vật thể, chính là YOLO (You Only Look One). \\
\vspace{0.5cm}
\noindent\textit{\textbf{Từ khóa:}} \textit{Deep Learning, Deep Neural Network, YOLO, Convolutional Neural Network.}
}