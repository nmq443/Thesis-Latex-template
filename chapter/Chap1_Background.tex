\clearpage
\phantomsection

\setcounter{chapter}{0}
\chapter[{TỔNG QUAN VỀ ĐỀ TÀI}]{TỔNG QUAN VỀ ĐỀ TÀI}

Trước khi tìm hiểu chi tiết về cơ sở lý thuyết hay các thuật toán trong Deep learning hoặc thuật toán YOLO thì ta cần tìm hiểu cơ bản về Deep learning.   

\section{Khái niệm cơ bản}

\subsection{Mạng neuron nhân tạo}
Mạng neuron nhân tạo (hay Artificial Neural Network) 

\subsection{Deep learning}
Deep Learning là một tập con của Machine learning (học máy), sử dụng mạng neuron nhân tạo với nhiều lớp (hay còn được gọi là mạng học sâu), lấy ý tưởng từ não bộ của chính con người để mô phỏng khả năng học và đưa ra quyết định một cách nhanh chóng và chính xác của con người. 

Định nghĩa một cách chính xác thì một mạng học sâu, hay deep neural network (DNN), là một mạng neuron nhân tạo với số lớp lớn hơn hoặc bằng 3. Trong các ứng dụng, mạng DNN thường có rất nhiều lớp. Mạng DNN được huấn luyện trên một bộ dữ liệu rất lớn để phân tích và nhận diện các đặc điểm, phân biệt hành vi và mối quan hệ, đánh giá xác suất, đưa ra dự đoán và quyết định. 

Tuy mạng neuron với một lớp có thể đưa ra những dự đoán và quyết định đủ tốt, việc thêm vào nhiều lớp hơn trong mạng neuron đó sẽ giúp tối ưu kết quả với độ chính xác cao hơn.

\newpage
\section{}

